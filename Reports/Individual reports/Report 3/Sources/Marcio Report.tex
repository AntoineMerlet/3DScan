\documentclass[aps,letterpaper,11pt]{revtex4}
\usepackage{graphicx} % For images
\usepackage{float}    % For tables and other floats
\usepackage{verbatim} % For comments and other
\usepackage{amssymb}  % For more math
\usepackage{fullpage} % Set margins and place page numbers at bottom center
\usepackage{listings} % For source code
\usepackage[usenames,dvipsnames]{color} % For colors and names
\usepackage[pdftex]{hyperref}           % For hyperlinks and indexing the PDF
\usepackage{pdfpages}
\usepackage{subfig}
\usepackage{listings}
\usepackage[usenames,dvipsnames,svgnames,table]{xcolor}
\usepackage{color}
\usepackage{textcomp}
\usepackage[utf8]{inputenc}
% Custom colors
\definecolor{deepblue}{rgb}{0,0,0.5}
\definecolor{deepred}{rgb}{0.6,0,0}
\definecolor{deepgreen}{rgb}{0,0.5,0}

 \lstset{
  tabsize=4,
  language=C++,
  captionpos=b,
  tabsize=3,
  numberstyle=\tiny,
  numbersep=5pt,
  breaklines=true,
  showstringspaces=false,
  basicstyle=\footnotesize,
%  identifierstyle=\color{magenta},
  keywordstyle=\color[rgb]{0,0,1},
  commentstyle=\color{deepgreen},
  stringstyle=\color{deepred}
  }
  
\hypersetup{ % play with the different link colors here
    colorlinks,
    citecolor=black,
    filecolor=black,
    linkcolor=black,
    urlcolor=blue % set to black to prevent printing blue links
}

\newcommand{\labno}{Software Engineering Project}
\newcommand{\labtitle}{Weekly Report}
\newcommand{\authorname}{Marcio Aloisio Bezerra Cavalcanti Rockenbach}
\newcommand{\professor}{Dr. Yohan Fougerolle, Dr. Cansen Jiang, Dr. David Strubel}


\begin{document}  
\begin{titlepage}
\begin{center}
{\LARGE \textsc{\labno:} \\ \vspace{4pt}}
{\Large \textsc{\labtitle} \\ \vspace{4pt}} 
\rule[13pt]{\textwidth}{1pt} \\ \vspace{150pt}
{\large By: \authorname \\ \vspace{10pt}
Professor: \professor \\ \vspace{10pt}
\today}
\end{center}


\end{titlepage}% END TITLE PAGE %%%%%%%%%%%%%%%%%%%%%%%%%%%%%%%%%%
\newpage
\setlength{\parindent}{5ex}

\section{Kinect Sensor}
Once again, I worked with the Kinect Sensor. I borrowed it from one of the students who are responsible for them (Eduardo Ochoa). My goal was to try to see if the function to save the point clouds as PCD was working after the changes I made on the 3D-Korn code. Unfortunately, the program still crashes if we try to save as PCD. The function to save as PLY, though, is still working.\\

\section{IO Research}
I did research on how to work with the IO tools with Point Cloud Library (PCL). I tried to run some simple examples on writing and reading PCL files given by the website, but I had very big problems running those codes. (\url{http://docs.pointclouds.org/trunk/group__io.html}).\\
\indent To input data from the Kinect to our project, we will make use of a class called kinect2Grabber, which implements data manipulation from the Kinect into PCL. This class, however, doesn't implement all of the functionalities we need to handle the project. For what I did on my research, I believe that some PCL classes will have to be used:\\
\indent - pcl::PCDReader\\
\indent - pcl::PCDWriter\\
\indent - pcl::PLYReader\\
\indent - pcl::PLYWriter\\
\indent - pcl::OpenNIGrabber\\
\indent - pcl::FileWriter - in that class, we can see that the function virtual int pcl::FileWriter::write has "orientation" as one of its parameters. This might be used to acquire the data in vertical orientation rather than in horizontal one, making it easier to manipulate the files.\\

\section{Next Steps}
In these next weeks, we will start coding and making improvements on the previous projects.\\
I'm concerned about how to implement those modifications, since appears to be a huge gap between what we learned in class and this project. As someone with no previous knowledge of C++, it takes a long time to understand even the most simple issues (how to install and include libraries, topics related to cmake, qmake, sensor interaction, file storage).
\\
\end{document} % DONE WITH DOCUMENT!

