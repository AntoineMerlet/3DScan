\documentclass[aps,letterpaper,11pt]{revtex4}
\usepackage{graphicx} % For images
\usepackage{float}    % For tables and other floats
\usepackage{verbatim} % For comments and other
\usepackage{amssymb}  % For more math
\usepackage{fullpage} % Set margins and place page numbers at bottom center
\usepackage{listings} % For source code
\usepackage[usenames,dvipsnames]{color} % For colors and names
\usepackage[pdftex]{hyperref}           % For hyperlinks and indexing the PDF
\usepackage{pdfpages}
\usepackage{subfig}
\usepackage{listings}
\usepackage[usenames,dvipsnames,svgnames,table]{xcolor}
\usepackage{color}
\usepackage{textcomp}
\usepackage[utf8]{inputenc}
% Custom colors
\definecolor{deepblue}{rgb}{0,0,0.5}
\definecolor{deepred}{rgb}{0.6,0,0}
\definecolor{deepgreen}{rgb}{0,0.5,0}

 \lstset{
  tabsize=4,
  language=C++,
  captionpos=b,
  tabsize=3,
  numberstyle=\tiny,
  numbersep=5pt,
  breaklines=true,
  showstringspaces=false,
  basicstyle=\footnotesize,
%  identifierstyle=\color{magenta},
  keywordstyle=\color[rgb]{0,0,1},
  commentstyle=\color{deepgreen},
  stringstyle=\color{deepred}
  }
  
\hypersetup{ % play with the different link colors here
    colorlinks,
    citecolor=black,
    filecolor=black,
    linkcolor=black,
    urlcolor=blue % set to black to prevent printing blue links
}

\newcommand{\labno}{Software Engineering Project}
\newcommand{\labtitle}{Weekly Report}
\newcommand{\authorname}{Marcio Aloisio Bezerra Cavalcanti Rockenbach}
\newcommand{\professor}{Dr. Yohan Fougerolle, Dr. Cansen Jiang, Dr. David Strubel}


\begin{document}  
\begin{titlepage}
\begin{center}
{\LARGE \textsc{\labno:} \\ \vspace{4pt}}
{\Large \textsc{\labtitle} \\ \vspace{4pt}} 
\rule[13pt]{\textwidth}{1pt} \\ \vspace{150pt}
{\large By: \authorname \\ \vspace{10pt}
Professor: \professor \\ \vspace{10pt}
\today}
\end{center}


\end{titlepage}% END TITLE PAGE %%%%%%%%%%%%%%%%%%%%%%%%%%%%%%%%%%
\newpage
\setlength{\parindent}{5ex}

\section{PCL Library Research}
I read documentations available at the PCL website to get familiar with the file format. I made some notes to summarize the most important topics. By understanding the file format, we are able to understand better how the software handles with the acquired data.\\

\section {Kinect Sensor}
Again this week, I went to the Lab to work with the Kinect Sensor. As decided by our group meeting, I was supposed to try to acquire data using the Kinect in vertical position instead of horizontal. With the help of Mladen, we acquired point clouds in that new format with the 3D Korn Project. Using their GUI, we were able to set the filters to acquire point clouds in the maximum size possible. By our observations, the results didn't differ much from the previous acquisition in the horizontal way. Additionally, the vertical acquisition requires changes in the post-processing to change the orientation of the produced MESH. Once again, we had problems in saving the point clouds in PCD format.\\

\begin{figure}[!htb]
\centering
\includegraphics[scale=0.3]{mladen_optimized.jpg}
\caption{3d Korn - Vertical Acqusition}
\end{figure}

\begin{figure}[!htb]
\centering
\includegraphics[scale=0.3]{mladen_vertical_mesh.jpg}
\caption{3D Korn - Vertical MESH}
\end{figure}


\section{Code Analysis}
I analysed the 3D Korn project to try to understand how they handle the data acquisition and storage and also to try to fix the PCD saving issue. By comparing the code about saving in PLY and PCD, I saw that the PCD part had some additional coding. I modified it, but I have to see if it works by working with the Kinect sensor again.\\
Also, I tried to fully understand the process of data acquisition in the previous project. The debugging is really difficult, mainly because of my lack of experience in programming and because the documentation is really poor. Several libraries and functions are used, but almost nothing is documented.\\
\section{Next Steps}
For the next week, I'm planning on working with the Kinect sensor again to try to save the point clouds in the PCD format. If the saving process is successful, I will try to read the PCD header to know exactly how the data is being stored.\\
I'm planning also on discussing with the group the best approach for the data acquisition  (horizontal x vertical).
\\
\end{document} % DONE WITH DOCUMENT!

