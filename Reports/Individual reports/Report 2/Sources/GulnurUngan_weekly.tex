\documentclass[aps,letterpaper,11pt]{revtex4}
\usepackage{graphicx} % For images
\usepackage{float}    % For tables and other floats
\usepackage{verbatim} % For comments and other
\usepackage{amssymb}  % For more math
\usepackage{fullpage} % Set margins and place page numbers at bottom center
\usepackage{listings} % For source code
\usepackage[usenames,dvipsnames]{color} % For colors and names
\usepackage[pdftex]{hyperref}           % For hyperlinks and indexing the PDF
\usepackage{pdfpages}
\usepackage{subfig}
\usepackage{listings}
\usepackage[usenames,dvipsnames,svgnames,table]{xcolor}
\usepackage{color}
\usepackage{textcomp}
\usepackage[utf8]{inputenc}
% Custom colors
\definecolor{deepblue}{rgb}{0,0,0.5}
\definecolor{deepred}{rgb}{0.6,0,0}
\definecolor{deepgreen}{rgb}{0,0.5,0}

 \lstset{
  tabsize=4,
  language=C++,
  captionpos=b,
  tabsize=3,
  numberstyle=\tiny,
  numbersep=5pt,
  breaklines=true,
  showstringspaces=false,
  basicstyle=\footnotesize,
%  identifierstyle=\color{magenta},
  keywordstyle=\color[rgb]{0,0,1},
  commentstyle=\color{deepgreen},
  stringstyle=\color{deepred}
  }
  
\hypersetup{ % play with the different link colors here
    colorlinks,
    citecolor=black,
    filecolor=black,
    linkcolor=black,
    urlcolor=blue % set to black to prevent printing blue links
}

\newcommand{\labno}{Software Engineering Project}
\newcommand{\labtitle}{Weekly Report}
\newcommand{\authorname}{Gulnur Semahat Ungan}
\newcommand{\professor}{Dr. Yohan Fougerolle, Dr. Cansen Jiang, Dr. David Strubel}
   

\begin{document}  
\begin{titlepage}
\begin{center}
{\LARGE \textsc{\labno:} \\ \vspace{4pt}}
{\Large \textsc{\labtitle} \\ \vspace{4pt}} 
\rule[13pt]{\textwidth}{1pt} \\ \vspace{150pt}
{\large By: \authorname \\ \vspace{10pt}
Professor: \professor \\ \vspace{10pt}
\today}
\end{center}




\end{titlepage}% END TITLE PAGE %%%%%%%%%%%%%%%%%%%%%%%%%%%%%%%%%%
\newpage

\section{Summary of the week}
Last meetin, I mentioned about my ideas for developing 3D project by improvement of ICP. I have collected so many articles about this issue. The followings happened : 

\begin{itemize}
\item KD tree algorithm is the most accurate and easy to implement.
\item But to see the ICP weakness, I wanted to see ICP approach on Matlab
\item After ICP, I would like to see the KD Tree result on Matlab
\item  I chose Matlab because , it is easier than C++ just for having an idea.
\item I will concentrate more about STL to do same things that I did on Matlab
\item I could not achieve to upload point cloud and apply KD Tree+ICP Matlab function.
\item During this week, I will fix the Matlab file and I would like to see the results in my screen.
\item It is obvious that, KD tree+ICP works very well.
\item  According to articles, The difference for Stanford Bunny is very valuable.
\item  To see the result on my computer, I used .ply files of 3D-Korn project. (project 1)
\item Althoug KD tree gives better result than naked ICP, I would like to concentrate GF+ICP algorithm according to one of article.
\item When I compare the power of algorithms IC,KD+ICP and GF+ICP, GF+ICP gives the best result among them.
\item  Next week, I will give the final decision about algorithm.
\item On the other hand, choosing filter can be another development .
\item I have found some articles about filtering 3D point cloud. Previous year projects while 1 used different filter algorithm, 2-3 and 4 used the same filters.
\item I think fresh idea about filtering can be a good improvement. Also to see the differences between filters, Using combination of filters can be good.
\item Until next week, I will have rough idea about filtering. First I have to focus on and finish the decision of registration algorithm
\item For KDTree+ICP algorithm, I just have seen the result of created object. I did not achieve arranging the Matlab file to see the result of uploaded 3DKorn point clouds.
\item I will try to modify Matlab code to see the results.
\end{itemize}

\begin{figure} [h]
\centering
\includegraphics [scale=.5] {rabi}
\caption{Difference between ICP,ICP+KD Tree and ICP+GF}
\end{figure} 

\begin{figure} [h]
\centering
\includegraphics [scale=.5] {1}
\caption{Matlab output from 3DKorn datas}
\end{figure} 


\begin{figure} [h]
\centering
\includegraphics [scale=.5] {2}
\caption{Matlab output from 3DKorn datas}
\end{figure} 

\begin{figure} [h]
\centering
\includegraphics [scale=.5] {3}
\caption{Matlab output from 3DKorn datas}
\end{figure} 


\end{document} % DONE WITH DOCUMENT!

