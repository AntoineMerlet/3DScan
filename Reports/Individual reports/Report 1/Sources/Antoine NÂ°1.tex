\documentclass[aps,letterpaper,11pt]{revtex4}
\usepackage{graphicx} % For images
\usepackage{float}    % For tables and other floats
\usepackage{verbatim} % For comments and other
\usepackage{amssymb}  % For more math
\usepackage{fullpage} % Set margins and place page numbers at bottom center
\usepackage{listings} % For source code
\usepackage[usenames,dvipsnames]{color} % For colors and names
\usepackage[pdftex]{hyperref}           % For hyperlinks and indexing the PDF
\usepackage{pdfpages}
\usepackage{subfig}
\usepackage{listings}
\usepackage[usenames,dvipsnames,svgnames,table]{xcolor}
\usepackage{color}
\usepackage{textcomp}
\usepackage[utf8]{inputenc}
% Custom colors
\definecolor{deepblue}{rgb}{0,0,0.5}
\definecolor{deepred}{rgb}{0.6,0,0}
\definecolor{deepgreen}{rgb}{0,0.5,0}

 \lstset{
  tabsize=4,
  language=C++,
  captionpos=b,
  tabsize=3,
  numberstyle=\tiny,
  numbersep=5pt,
  breaklines=true,
  showstringspaces=false,
  basicstyle=\footnotesize,
%  identifierstyle=\color{magenta},
  keywordstyle=\color[rgb]{0,0,1},
  commentstyle=\color{deepgreen},
  stringstyle=\color{deepred}
  }
  
\hypersetup{ % play with the different link colors here
    colorlinks,
    citecolor=black,
    filecolor=black,
    linkcolor=black,
    urlcolor=blue % set to black to prevent printing blue links
}

\newcommand{\labno}{Personal Report N.1}
\newcommand{\labtitle}{Software Engineering Project}
\newcommand{\authorname}{Antoine Merlet}
\newcommand{\professor}{Dr. Yohan Fougerolle, Dr. Cansen Jiang, Dr. David Strubel}


\begin{document}  
\begin{titlepage}
\begin{center}
{\LARGE \textsc{\labno:} \\ \vspace{4pt}}
{\Large \textsc{\labtitle} \\ \vspace{4pt}} 
\rule[13pt]{\textwidth}{1pt} \\ \vspace{150pt}
{\large By: \authorname \\ \vspace{10pt}
Professor: \professor \\ \vspace{10pt}
\today}
\end{center}




\end{titlepage}% END TITLE PAGE %%%%%%%%%%%%%%%%%%%%%%%%%%%%%%%%%%
\newpage

This is my first report. It came pretty late according to the beginning of this project, and the reasons will be detailed in this report.


\section{Team management}
From our backgrounds, we determined that I had the most knowledge regarding Software development and design. From this, I inherite the following taks:

\begin{itemize}
\item SVN management (Github)
\item Dependencies installation
\item Software design
\item Coding conventions management (documentation, management)
\item Team meeting leading
\end{itemize}

On top of improving our knowledge of C++ (which is the main goal of this project), i will try to share my knowledge about good habits in coding and software engineering.

\section{Software installation}

I had the task to manage  the installation part of the project, ie. figuring a way to run the previous codes. Even if each report from the previous year tried to explain how to do so, their guides were not sufficent to allow an easy installation (due to outdates links, missing informations, etc). This part took me a lot of time in the past weeks, as we did not have any knowledge of Cmake, Qmake, .pro files, libraries, etc... The other teams had troubles as well, and we shared our experience and solutions to make it work. The problems were library linking and .dll locating, but mostly QVTK generation. All this problem were solved first by our classmate Roger, who kindly wrote an installation guide, and provided the files if needed.
For this project, we will provide, on a File storage platform, all the required compenents, for and easy installatiion.

\section{Code study}

Here is a quick review of the current projects:

\begin{itemize}
\item 3D-Korn: doc, comments, classes, complete GUI.
\item 3D-Scanner-GUI: No OOP, no documentation, few comments, strange GUI implementation choice
\item the-scanner: inconsistent doc, OOP, interesting GUI with settings management
\item vtk: manually created GUI, few comment/doc.
\end{itemize}

Overall, The projects 3D-Korn and the-scanner might be easy to reuse, due to their organization and the ideas they bring.


\section{Future plan}

For next week, I will produce a meaningful UML class diagram for this project. After validation with the team, it will be filled with class attribute members and functions. We will then be able to start coding.

\end{document} % DONE WITH DOCUMENT!

