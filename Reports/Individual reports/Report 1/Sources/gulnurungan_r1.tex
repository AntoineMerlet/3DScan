\documentclass[aps,letterpaper,11pt]{revtex4}
\usepackage{graphicx} % For images
\usepackage{float}    % For tables and other floats
\usepackage{verbatim} % For comments and other
\usepackage{amssymb}  % For more math
\usepackage{fullpage} % Set margins and place page numbers at bottom center
\usepackage{listings} % For source code
\usepackage[usenames,dvipsnames]{color} % For colors and names
\usepackage[pdftex]{hyperref}           % For hyperlinks and indexing the PDF
\usepackage{pdfpages}
\usepackage{subfig}
\usepackage{listings}
\usepackage[usenames,dvipsnames,svgnames,table]{xcolor}
\usepackage{color}
\usepackage{textcomp}
\usepackage[utf8]{inputenc}
% Custom colors
\definecolor{deepblue}{rgb}{0,0,0.5}
\definecolor{deepred}{rgb}{0.6,0,0}
\definecolor{deepgreen}{rgb}{0,0.5,0}

 \lstset{
  tabsize=4,
  language=C++,
  captionpos=b,
  tabsize=3,
  numberstyle=\tiny,
  numbersep=5pt,
  breaklines=true,
  showstringspaces=false,
  basicstyle=\footnotesize,
%  identifierstyle=\color{magenta},
  keywordstyle=\color[rgb]{0,0,1},
  commentstyle=\color{deepgreen},
  stringstyle=\color{deepred}
  }
  
\hypersetup{ % play with the different link colors here
    colorlinks,
    citecolor=black,
    filecolor=black,
    linkcolor=black,
    urlcolor=blue % set to black to prevent printing blue links
}

\newcommand{\labno}{Weekly Report-26/11/2017}
\newcommand{\labtitle}{Software Engineering Project-3D Scan}
\newcommand{\authorname}{Gulnur Semahat Ungan}
\newcommand{\professor}{Dr. Yohan Fougerolle, Dr. Cansen Jiang, Dr. David Strubel}
   

\begin{document}  
\begin{titlepage}
\begin{center}
{\LARGE \textsc{\labno:} \\ \vspace{4pt}}
{\Large \textsc{\labtitle} \\ \vspace{4pt}} 
\rule[13pt]{\textwidth}{1pt} \\ \vspace{150pt}
{\large By: \authorname \\ \vspace{10pt}
Professor: \professor \\ \vspace{10pt}
\today}
\end{center}




\end{titlepage}% END TITLE PAGE %%%%%%%%%%%%%%%%%%%%%%%%%%%%%%%%%%
\newpage

\section{SUMMARY OF THE WEEK}
After meeting and decentralisation ;I took care of math part of this project. 
Math part has the following concept and problems to be solved:

\begin{itemize}
\item To get warm the concept of ICP
\item Try to understand the previous project's ICP concept(from reports or source codes)
\item To read the articles which are available at Edmodo.
\item To open the previous projects and see the results.
\item To find a more advanced solution for the projects as development of them
\end{itemize}
For the challenge of opening the project on my computer. I just tried to open first group's project. (3D-Korn). I realized that I have to work on PCL to understand the codes but just for coarsely into applied ICP approach to the project, I believe that ICP part should be improved. Thats why I found some different articles about improved iterative closest point algorithm. I have just skimmed these articles; it is obvious that these developments can be seen depending upon the number of data for a body. Actually, first calculating as just for having opinion; number is satisfying to apply some developed algorithms. This issue has to be thought deeply. It will be done previous days.

\end{document} % DONE WITH DOCUMENT!

