\documentclass[aps,letterpaper,11pt]{revtex4}
\usepackage{graphicx} % For images
\usepackage{float}    % For tables and other floats
\usepackage{verbatim} % For comments and other
\usepackage{amssymb}  % For more math
\usepackage{fullpage} % Set margins and place page numbers at bottom center
\usepackage{listings} % For source code
\usepackage[usenames,dvipsnames]{color} % For colors and names
\usepackage[pdftex]{hyperref}           % For hyperlinks and indexing the PDF
\usepackage{pdfpages}
\usepackage{subfig}
\usepackage{listings}
\usepackage[usenames,dvipsnames,svgnames,table]{xcolor}
\usepackage{color}
\usepackage{textcomp}
\usepackage[utf8]{inputenc}
% Custom colors
\definecolor{deepblue}{rgb}{0,0,0.5}
\definecolor{deepred}{rgb}{0.6,0,0}
\definecolor{deepgreen}{rgb}{0,0.5,0}

 \lstset{
  tabsize=4,
  language=C++,
  captionpos=b,
  tabsize=3,
  numberstyle=\tiny,
  numbersep=5pt,
  breaklines=true,
  showstringspaces=false,
  basicstyle=\footnotesize,
%  identifierstyle=\color{magenta},
  keywordstyle=\color[rgb]{0,0,1},
  commentstyle=\color{deepgreen},
  stringstyle=\color{deepred}
  }
  
\hypersetup{ % play with the different link colors here
    colorlinks,
    citecolor=black,
    filecolor=black,
    linkcolor=black,
    urlcolor=blue % set to black to prevent printing blue links
}

\newcommand{\labno}{Software Engineering Project}
\newcommand{\labtitle}{Weekly report}
\newcommand{\authorname}{Mladen Rakic}
\newcommand{\professor}{Dr. Yohan Fougerolle}


\begin{document}  
\begin{titlepage}
\begin{center}
{\LARGE \textsc{\labno:} \\ \vspace{4pt}}
{\Large \textsc{\labtitle} \\ \vspace{4pt}} 
\rule[13pt]{\textwidth}{1pt} \\ \vspace{150pt}
{\large By: \authorname \\ \vspace{10pt}
Professor: \professor \\ \vspace{10pt}
\today}
\end{center}


\end{titlepage}% END TITLE PAGE %%%%%%%%%%%%%%%%%%%%%%%%%%%%%%%%%%
\newpage
\section {Introduction}
This is the first weekly report in which I will make a brief description of the things I have managed to achieve up to this point. It will list all the things I did in the previous several weeks, as well as the short preview of the future tasks.\\
\section {Accomplished tasks}
This is the brief list of the things I have managed to accomplish so far:\\
\linebreak
1. I have read the reports from the previous year's students in order to understand the project structure and goals. As I was reading them, I tried to focus on things that might be improved in our project. My specific point of interest is the GUI, since the group agreed that I should work on that mainly. In the weeks to follow I will try to analyse and improve the GUIs developed by the previous year's students.\\ 
\linebreak 
2. I read the papers Mr Cansen Jiang posted on Edmodo platform to understand the key concepts required for the implementation of the project. I tried to get familiar with the maths behind the algorithm and to understand which libraries we should use for the project and why.\\ 
\linebreak
3. In order to manage the project properly, I had to get familiar with the basic concepts of Git and GitHub and in order to document the progress I am getting used to using Latex.\\
\linebreak
NOTE: Since the OS on my computer is Windows 7, I was not able to run the previous year's projects. That is because of the incompatibility of Windows 7 with Kinect V2. Because of that, we have come to an agreement that I will be using Marcio's notebook for the coding part. \\
\linebreak
4. That being said, we managed to run the previous projects and to take a look at the source codes.\\
\linebreak
5. Also, we succeeded in connecting the Kinect V2 to the computer and in acquiring some raw data (fig. 1 and fig. 2). However, when we tried to save some data for further use, the computer crashed. That may be because of the format used to save the data. We will try to change that and do the whole process again during next week, but for now it is left to be further explored.\\

\begin{figure}[!htb]
  \includegraphics[scale=0.3]{Kinect2.jpg}
  \caption{Data acquisition using Kinect Studio.}
  \label{fig:Kinect2}
\end{figure}

\begin{figure}[!htb]
  \includegraphics[scale=0.3]{MladenCloud.jpg}
  \caption{Data acquisition using one of previous year's projects.}
  \label{fig:MladenCloud}
\end{figure}

6. Prior to testing the Kinect sensor and the acquisition of the data we had to get familiar with the sensor itself. For that reason, we watched some video tutorials explaining the way it works and how to obtain the data.\\
\section {Future tasks}
Here I will present the list of my assignments for the upcoming week(s):\\
\linebreak
1. First of all, since I do not have a significant background in computer science and software engineering, and most of the concepts are brand new to me, I will have to get more familiar with the source codes (mainly with the structure and organisation).\\
\linebreak
2. As previously said, I will be the one focused on the improvement and development of the GUI for our project. I chose to do that because I never had the chance to implement one, so this will be the opportunity to tackle that challenge.\\
\linebreak
3. Lastly, for the more general group task, and not the individual one, we will try to make a detailed schedule and the organization of work in the next couple of days.  


\end{document} % DONE WITH DOCUMENT!

