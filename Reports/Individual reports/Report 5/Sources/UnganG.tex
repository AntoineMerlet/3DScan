\documentclass[aps,letterpaper,11pt]{revtex4}
\usepackage{graphicx} % For images
\usepackage{float}    % For tables and other floats
\usepackage{verbatim} % For comments and other
\usepackage{amssymb}  % For more math
\usepackage{fullpage} % Set margins and place page numbers at bottom center
\usepackage{listings} % For source code
\usepackage[usenames,dvipsnames]{color} % For colors and names
\usepackage[pdftex]{hyperref}           % For hyperlinks and indexing the PDF
\usepackage{pdfpages}
\usepackage{subfig}
\usepackage{listings}
\usepackage[usenames,dvipsnames,svgnames,table]{xcolor}
\usepackage{color}
\usepackage{textcomp}
\usepackage[utf8]{inputenc}
% Custom colors
\definecolor{deepblue}{rgb}{0,0,0.5}
\definecolor{deepred}{rgb}{0.6,0,0}
\definecolor{deepgreen}{rgb}{0,0.5,0}

 \lstset{
  tabsize=4,
  language=C++,
  captionpos=b,
  tabsize=3,
  numberstyle=\tiny,
  numbersep=5pt,
  breaklines=true,
  showstringspaces=false,
  basicstyle=\footnotesize,
%  identifierstyle=\color{magenta},
  keywordstyle=\color[rgb]{0,0,1},
  commentstyle=\color{deepgreen},
  stringstyle=\color{deepred}
  }
  
\hypersetup{ % play with the different link colors here
    colorlinks,
    citecolor=black,
    filecolor=black,
    linkcolor=black,
    urlcolor=blue % set to black to prevent printing blue links
}

\newcommand{\labno}{Software Engineering Project}
\newcommand{\labtitle}{Weekly Reports-For 2 Weeks}
\newcommand{\authorname}{Gulnur Semahat Ungan}
\newcommand{\professor}{Dr. Yohan Fougerolle, Dr. Cansen Jiang, Dr. David Strubel}
   

\begin{document}  
\begin{titlepage}
\begin{center}
{\LARGE \textsc{\labno:} \\ \vspace{4pt}}
{\Large \textsc{\labtitle} \\ \vspace{4pt}} 
\rule[13pt]{\textwidth}{1pt} \\ \vspace{150pt}
{\large By: \authorname \\ \vspace{10pt}
Professor: \professor \\ \vspace{10pt}
\today}
\end{center}




\end{titlepage}% END TITLE PAGE %%%%%%%%%%%%%%%%%%%%%%%%%%%%%%%%%%
\newpage

\section{Summary of the week}
Last week, because of final week ,I could not study on this project. This week, I tried to figure out descriptor of one on feature based method for point cloud registration. k-nearest neighbour algorithm. To use and see this result I checked https://github.com/gvd/kdtree algorithm. I will work on the article that I uploaded 2 weeks ago. I hope I will implement it as soon as possible. Registration part includes not only cpp knowledge but also some mathematical talent to understand and to convert it into a language such as Cpp. According to this article there are 3 tasks to be done. First one is to feature selection method. Second one is to find local neighbours of all points in a point cloud efficiently. Third one is to find corresponding points in Point Clouds being registreted.  Also under three situations this method worked well. These are : (M and P are  point clouds ) (1)M and P are PCs of the whole model. (2) M is the PC of a whole model and P is a randomly selected subset of M containing 1/4 of points in M. (3) M and P have overlapping regions of varying ratio, but P is not a subset of M.


\end{document} % DONE WITH DOCUMENT!
