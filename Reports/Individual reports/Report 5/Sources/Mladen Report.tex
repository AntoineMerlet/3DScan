\documentclass[aps,letterpaper,11pt]{revtex4}
\usepackage{graphicx} % For images
\usepackage{float}    % For tables and other floats
\usepackage{verbatim} % For comments and other
\usepackage{amssymb}  % For more math
\usepackage{fullpage} % Set margins and place page numbers at bottom center
\usepackage{listings} % For source code
\usepackage[usenames,dvipsnames]{color} % For colors and names
\usepackage[pdftex]{hyperref}           % For hyperlinks and indexing the PDF
\usepackage{pdfpages}
\usepackage{subfig}
\usepackage{listings}
\usepackage[usenames,dvipsnames,svgnames,table]{xcolor}
\usepackage{color}
\usepackage{textcomp}
\usepackage[utf8]{inputenc}
% Custom colors
\definecolor{deepblue}{rgb}{0,0,0.5}
\definecolor{deepred}{rgb}{0.6,0,0}
\definecolor{deepgreen}{rgb}{0,0.5,0}

 \lstset{
  tabsize=4,
  language=C++,
  captionpos=b,
  tabsize=3,
  numberstyle=\tiny,
  numbersep=5pt,
  breaklines=true,
  showstringspaces=false,
  basicstyle=\footnotesize,
%  identifierstyle=\color{magenta},
  keywordstyle=\color[rgb]{0,0,1},
  commentstyle=\color{deepgreen},
  stringstyle=\color{deepred}
  }
  
\hypersetup{ % play with the different link colors here
    colorlinks,
    citecolor=black,
    filecolor=black,
    linkcolor=black,
    urlcolor=blue % set to black to prevent printing blue links
}

\newcommand{\labno}{Software Engineering Project}
\newcommand{\labtitle}{Weekly report}
\newcommand{\authorname}{Mladen Rakic}
\newcommand{\professor}{Dr. Yohan Fougerolle}


\begin{document}  
\begin{titlepage}
\begin{center}
{\LARGE \textsc{\labno:} \\ \vspace{4pt}}
{\Large \textsc{\labtitle} \\ \vspace{4pt}} 
\rule[13pt]{\textwidth}{1pt} \\ \vspace{150pt}
{\large By: \authorname \\ \vspace{10pt}
Professor: \professor \\ \vspace{10pt}
\today}
\end{center}


\end{titlepage}% END TITLE PAGE %%%%%%%%%%%%%%%%%%%%%%%%%%%%%%%%%%
\newpage
\section {Accomplished tasks}
This is the brief list of the things I have managed to accomplish this week:\\
\linebreak
1. I managed to improve the GUI design according to the modifications proposed by the group. Those include managing the coordinates in a more user friendly manner (using sliders), putting the logger in the GUI in order to be able to keep track of the progress when particular things are executed, as well as the possibility to choose between horizontal and vertical acquisition.\\ 
\linebreak 
2. The current design of the GUI is missing potentially only one thing, which is the possibility to choose among several proposed registration methods. However, it is still to be seen which ones we will fully include in the project. \\ 
\linebreak
3. I started coding the GUI and connecting the components, but for most of the things I am still missing some input from the group. However, I wanted to allow the display and the visualization of the point clouds, so that we can test the registration algorithms. The problem that occurred was that I am missing QVTKWidget component on my machine. I tried to figure out how to solve this and asked Marcio for help, but we were not able to solve this issue so far.\\
\pagebreak

\section {Future tasks}
Here I will present the list of my assignments for the upcoming week(s):\\
\linebreak
1. First and most important thing to do now is to resolve the issue with the visualization and QVTKWidget. I will try to work on it on my own, to test several more options, and if I am not successful, 
I will look out for help elsewhere.\\
\linebreak
2. After that is done, I will wait to get more information from the rest of the group in order to be able to link everything properly in the GUI. \\

\end{document} % DONE WITH DOCUMENT!

